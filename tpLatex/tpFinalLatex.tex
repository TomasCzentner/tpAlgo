\documentclass[spanish,a4paper]{article}

\usepackage[spanish]{babel}
\selectlanguage{spanish}
\usepackage[utf8]{inputenc}
%\usepackage{bbm}
\usepackage{framed}
% Paquetes generales
\usepackage{ifthen}
\usepackage{amssymb}
\usepackage{multicol}
\usepackage[absolute]{textpos}

%%%%%%%%%%%%%%%%%%%%%%%% TWEAKLIST.STY %%%%%%%%%%%%%%%%%%%%%%%%
%% Esto esta copiado de tweaklist.sty, un paquete que encontré en la
%% web. Define hooks que se ejecutan cada vez que se invoca
%% determinado environment (itemhook, enumhook y descripthook
%%%%%%%%%%%%%%%%%%%%%%%%%%%%%%%%%%%%%%%%%%%%%%%%%%%%%%%%%%%%%%%
\makeatletter
\def\enumhook{}
% \def\enumhooki{}\def\enumhookii{}\def\enumhookiii{}
% \def\enumhookiv{}\def\itemhook{}\def\itemhooki{}\def\itemhookii{}
% \def\itemhookiii{}\def\itemhookiv{}\def\descripthook{}
\def\enumerate{%
  \ifnum \@enumdepth >\thr@@\@toodeep\else
    \advance\@enumdepth\@ne
    \edef\@enumctr{enum\romannumeral\the\@enumdepth}%
      \expandafter
      \list
        \csname label\@enumctr\endcsname
        {\usecounter\@enumctr\def\makelabel##1{\hss\llap{##1}}%
          \enumhook \csname enumhook\romannumeral\the\@enumdepth\endcsname}%
  \fi}
% \def\itemize{%
%   \ifnum \@itemdepth >\thr@@\@toodeep\else
%     \advance\@itemdepth\@ne
%     \edef\@itemitem{labelitem\romannumeral\the\@itemdepth}%
%     \expandafter
%     \list
%       \csname\@itemitem\endcsname
%       {\def\makelabel##1{\hss\llap{##1}}%
%         \itemhook \csname itemhook\romannumeral\the\@itemdepth\endcsname}%
%   \fi}
% \renewenvironment{description}
%                  {\list{}{\labelwidth\z@ \itemindent-\leftmargin
%                           \let\makelabel\descriptionlabel\descripthook}}
%                  {\endlist}
%%%%%%%%%%%%%%%%%%%%%%%% TWEAKLIST.STY %%%%%%%%%%%%%%%%%%%%%%%%


% En las practicas usamos numeros arabigos para los ejercicios.
% Aca cambiamos los enumerate comunes para que usen letras y numeros
% romanos
\newcommand{\arreglarincisos}{%
  \renewcommand{\theenumi}{\alph{enumi}}
  \renewcommand{\theenumii}{\roman{enumii}}
  \renewcommand{\labelenumi}{\theenumi)}
  \renewcommand{\labelenumii}{\theenumii)}
}

%%%%%%%%%%%%%%%%%%%%%%%%%%%%%% PARCIAL %%%%%%%%%%%%%%%%%%%%%%%%
\let\@xa\expandafter
\newcommand{\tituloparcial}{\centerline{\depto -- \lamateria}
  \centerline{\elnombre -- \lafecha}%
  \setlength{\TPHorizModule}{10mm} % Fija las unidades de textpos
  \setlength{\TPVertModule}{\TPHorizModule} % Fija las unidades de
                                % textpos
  \arreglarincisos
  \newcounter{total}% Este contador va a guardar cuantos incisos hay
                    % en el parcial. Si un ejercicio no tiene incisos,
                    % cuenta como un inciso.
  \newcounter{contgrilla} % Para hacer ciclos
  \newcounter{columnainicial} % Se van a usar para los cline cuando un
  \newcounter{columnafinal}   % ejercicio tenga incisos.
  \newcommand{\primerafila}{}
  \newcommand{\segundafila}{}
  \newcommand{\rayitas}{} % Esto va a guardar los \cline de los
                          % ejercicios con incisos, asi queda mas bonito
  \newcommand{\anchodegrilla}{20} % Es para textpos
  \newcommand{\izquierda}{7} % Estos dos le dicen a textpos donde colocar
  \newcommand{\abajo}{2}     % la grilla
  \newcommand{\anchodecasilla}{0.4cm}
  \setcounter{columnainicial}{1}
  \setcounter{total}{0}
  \newcounter{ejercicio}
  \setcounter{ejercicio}{0}
  \newenvironment{ejercicio}[1]
  {%
    \stepcounter{ejercicio}\textbf{Ejercicio \theejercicio. [##1
      puntos]}% Formato
    \renewcommand\@currentlabel{\theejercicio}% Esto es para las
                                % referencias
    \newcommand{\invariante}[2]{%
      {\normalfont\bfseries\ttfamily invariante}%
      \ ####1\hspace{1em}####2%
    }%
    \renewcommand{\problema}[5][result]{
      \encabezadoDeProblema{####1}{####2}{####3}{####4}\hspace{1em}####5}%
  }% Aca se termina el principio del ejercicio
  {% Ahora viene el final
    % Esto suma la cantidad de incisos o 1 si no hubo ninguno
    \ifthenelse{\equal{\value{enumi}}{0}}
    {\addtocounter{total}{1}}
    {\addtocounter{total}{\value{enumi}}}
    \ifthenelse{\equal{\value{ejercicio}}{1}}{}
    {
      \g@addto@macro\primerafila{&} % Si no estoy en el primer ej.
      \g@addto@macro\segundafila{&}
    }
    \ifthenelse{\equal{\value{enumi}}{0}}
    {% No tiene incisos
      \g@addto@macro\primerafila{\multicolumn{1}{|c|}}
      \bgroup% avoid overwriting somebody else's value of \tmp@a
      \protected@edef\tmp@a{\theejercicio}% expand as far as we can
      \@xa\g@addto@macro\@xa\primerafila\@xa{\tmp@a}%
      \egroup% restore old value of \tmp@a, effect of \g@addto.. is
      
      \stepcounter{columnainicial}
    }
    {% Tiene incisos
      % Primero ponemos el encabezado
      \g@addto@macro\primerafila{\multicolumn}% Ahora el numero de items
      \bgroup% avoid overwriting somebody else's value of \tmp@a
      \protected@edef\tmp@a{\arabic{enumi}}% expand as far as we can
      \@xa\g@addto@macro\@xa\primerafila\@xa{\tmp@a}%
      \egroup% restore old value of \tmp@a, effect of \g@addto.. is
      % global 
      % Ahora el formato
      \g@addto@macro\primerafila{{|c|}}%
      % Ahora el numero de ejercicio
      \bgroup% avoid overwriting somebody else's value of \tmp@a
      \protected@edef\tmp@a{\theejercicio}% expand as far as we can
      \@xa\g@addto@macro\@xa\primerafila\@xa{\tmp@a}%
      \egroup% restore old value of \tmp@a, effect of \g@addto.. is
      % global 
      % Ahora armamos la segunda fila
      \g@addto@macro\segundafila{\multicolumn{1}{|c|}{a}}%
      \setcounter{contgrilla}{1}
      \whiledo{\value{contgrilla}<\value{enumi}}
      {%
        \stepcounter{contgrilla}
        \g@addto@macro\segundafila{&\multicolumn{1}{|c|}}
        \bgroup% avoid overwriting somebody else's value of \tmp@a
        \protected@edef\tmp@a{\alph{contgrilla}}% expand as far as we can
        \@xa\g@addto@macro\@xa\segundafila\@xa{\tmp@a}%
        \egroup% restore old value of \tmp@a, effect of \g@addto.. is
        % global 
      }
      % Ahora armo las rayitas
      \setcounter{columnafinal}{\value{columnainicial}}
      \addtocounter{columnafinal}{-1}
      \addtocounter{columnafinal}{\value{enumi}}
      \bgroup% avoid overwriting somebody else's value of \tmp@a
      \protected@edef\tmp@a{\noexpand\cline{%
          \thecolumnainicial-\thecolumnafinal}}%
      \@xa\g@addto@macro\@xa\rayitas\@xa{\tmp@a}%
      \egroup% restore old value of \tmp@a, effect of \g@addto.. is
      \setcounter{columnainicial}{\value{columnafinal}}
      \stepcounter{columnainicial}
    }
    \setcounter{enumi}{0}%
    \vspace{0.2cm}%
  }%
  \newcommand{\tercerafila}{}
  \newcommand{\armartercerafila}{
    \setcounter{contgrilla}{1}
    \whiledo{\value{contgrilla}<\value{total}}
    {\stepcounter{contgrilla}\g@addto@macro\tercerafila{&}}
  }
  \newcommand{\grilla}{%
    \g@addto@macro\primerafila{&\textbf{TOTAL}}
    \g@addto@macro\segundafila{&}
    \g@addto@macro\tercerafila{&}
    \armartercerafila
    \ifthenelse{\equal{\value{total}}{\value{ejercicio}}}
    {% No hubo incisos
      \begin{textblock}{\anchodegrilla}(\izquierda,\abajo)
        \begin{tabular}{|*{\value{total}}{p{\anchodecasilla}|}c|}
          \hline
          \primerafila\\
          \hline
          \tercerafila\\
          \tercerafila\\
          \hline
        \end{tabular}
      \end{textblock}
    }
    {% Hubo incisos
      \begin{textblock}{\anchodegrilla}(\izquierda,\abajo)
        \begin{tabular}{|*{\value{total}}{p{\anchodecasilla}|}c|}
          \hline
          \primerafila\\
          \rayitas
          \segundafila\\
          \hline
          \tercerafila\\
          \tercerafila\\
          \hline
        \end{tabular}
      \end{textblock}
    }
  }%
  \vspace{0.4cm}
  \textbf{LU:}
  
  \textbf{Apellidos:}
  
  \textbf{Nombres:}
  \vspace{0.5cm}
}
%%%%%%%%%%%%%%%%%%%%%%%%%%%%%% PARCIAL %%%%%%%%%%%%%%%%%%%%%%%%

% Esta parte arma cosas que dependen de si estamos usando beamer o no
% tocarEspacios ajusta leftskip y parindent para poder usarlas 
\@ifclassloaded{beamer}{%
  \newcommand{\tocarEspacios}{%
    \addtolength{\leftskip}{4em}%
    \addtolength{\parindent}{-3em}%
  }%
}
{%
  \usepackage[top=1cm,bottom=2cm,left=1cm,right=1cm]{geometry}%
  \usepackage{color}%
  \newcommand{\tocarEspacios}{%
    \addtolength{\leftskip}{2.5em}%
    \addtolength{\parindent}{-3em}%
  }%
}

\@ifundefined{mod}{%
  \newcommand{\mod}{\ \nom{mod}\ }%
}{%
  \renewcommand{\mod}{\ \nom{mod}\ }%
}
% Simbolos varios

% La Z de los numeros enteros
\newcommand{\ent}{\ensuremath{\mathbb{Z}}}
% La R de float
\newcommand{\float}{\ensuremath{\mathbb{R}}}
% El tipo Bool
\newcommand{\bool}{\ensuremath{\mathsf{Bool}}}
\newcommand{\True}{\ensuremath{\mathrm{True}}}
\newcommand{\False}{\ensuremath{\mathrm{False}}}
\newcommand{\Then}{\ensuremath{\rightarrow}}
\newcommand{\Iff}{\ensuremath{\leftrightarrow}}
\newcommand{\implica}{\ensuremath{\longrightarrow}}
\newcommand{\IfThenElse}[3]{\ensuremath{\mathsf{if}\ #1\ \mathsf{then}\ #2\ \mathsf{else}\ #3}}

% Comandos de formato
\newcommand{\nom}[1]{\ensuremath{\mathsf{#1}}}
% Comando para un comentario entre /* */. Font normal
\newcommand{\comentario}[1]{{/*\ #1\ */}}

\newcommand{\ya}{ya especificado en el trabajo anterior}

% Comandos del lenguaje de especificacion
% Selector para sacar elementos de una lista
\newcommand{\selec}{\ensuremath{\leftarrow}}
% La lista vacia
\newcommand{\lv}{[\,]}
% El ++ "bonito"
\newcommand{\masmas}{\ensuremath{+\!\!\!+}}

% Las barritas
\newcommand{\longitud}[1]{\ensuremath{\left|#1\right|}}
\newcommand{\cons}{\nom{cons}}
\newcommand{\indice}{\nom{indice}}
\newcommand{\conc}{\nom{conc}}
\newcommand{\concat}{\nom{concat}}
\newcommand{\cab}{\nom{cab}}
\newcommand{\cola}{\nom{cola}}
\newcommand{\sub}{\nom{sub}}
\newcommand{\en}{\nom{en}}
\newcommand{\cuenta}[2]{\nom{cuenta}\ensuremath{(#1, #2)}}
\newcommand{\suma}{\nom{suma}}
\newcommand{\twodots}{\nom{..}}
\newcommand{\rango}[2]{[#1\twodots#2]}
\newcommand{\rangoac}[2]{(#1\twodots#2]}
\newcommand{\rangoca}[2]{[#1\twodots#2)}
\newcommand{\rangoaa}[2]{(#1\twodots#2)}


% Listas por comprension. El primer parametro es la expresion y el
% segundo tiene los selectores y las condiciones.
\newcommand{\comp}[2]{[\,#1\,|\,#2\,]}
% Listas por extensión
\newcommand{\ext}[1]{[\,#1\,]}

% acum: el primer parametro es la expresion, el segundo la definicion
% de la variable de acumulacion, y el tercero los selectores y condiciones.
\newcommand{\acum}[3]{\mathrm{acum}(#1\; | \; #2, #3)}

\newcommand{\sinonimo}[2]{%
  \noindent%
  {\normalfont\bfseries\ttfamily tipo\ }%
  #1\ =\ #2%
  {\normalfont\bfseries\,;\par}
}

\newcommand{\tupla}[2]{\ensuremath{\langle}#1, #2\ensuremath{\rangle}}

% El primer parámetro es el nombre del tipo
% El segundo parámetro es la lista de elementos
\newcommand{\enumerado}[2]{%
  \noindent%
  {\normalfont\bfseries\ttfamily tipo\ }%
  #1\ =\ #2%
  {\normalfont\bfseries\,;\par}
}

\newcommand{\aux}[4]{%
  {
    \addtolength{\leftskip}{1em}
    \addtolength{\parindent}{-2.5em}
    {\normalfont\bfseries\ttfamily aux\ }%
    {\normalfont\ttfamily #1}%
    \ifthenelse{\equal{#2}{}}{}{\ (#2)}\ : #3\, = \ensuremath{#4}%
    {\normalfont\bfseries\,;\par}
  }
}

\newcommand{\encabezadoDeProblema}[4]{%
  % Ponemos la palabrita problema en tt
%  \noindent%
  {\normalfont\bfseries\ttfamily problema}%
  % Ponemos el nombre del problema
  \ %
  {\normalfont\ttfamily #2}%
  \ 
  % Ponemos los parametros
  (#3)%
  \ifthenelse{\equal{#4}{}}{}{%
  \ =\ %
  % Ponemos el nombre del resultado
  {\normalfont\ttfamily #1}%
  % Por ultimo, va el tipo del resultado
  \ : #4}
}

\newcommand{\encabezadoDeTipo}[2]{%
  % Ponemos la palabrita tipo en tt
  {\normalfont\bfseries\ttfamily tipo}%
  % Ponemos el nombre del tipo
  \ %
  {\normalfont\ttfamily #2}%
  \ifthenelse{\equal{#1}{}}{}{$\langle$#1$\rangle$}
}

\newenvironment{problema}[4][result]{%
  % El parametro 1 (opcional) es el nombre del resultado
  % El parametro 2 es el nombre del problema
  % El parametro 3 son los parametros
  % El parametro 4 es el tipo del resultado
  % Preambulo del ambiente problema
  % Tenemos que definir los comandos requiere, asegura, modifica y aux
  \newcommand{\requiere}[2][]{%
    {\normalfont\bfseries\ttfamily requiere}%
    \ifthenelse{\equal{##1}{}}{}{\ {\normalfont\ttfamily ##1} :}\ %
    \ensuremath{##2}%
    {\normalfont\bfseries\,;\par}%
  }
  \newcommand{\asegura}[2][]{%
    {\normalfont\bfseries\ttfamily asegura}%
    \ifthenelse{\equal{##1}{}}{}{\ {\normalfont\ttfamily ##1} :}\
    \ensuremath{##2}%
    {\normalfont\bfseries\,;\par}%
  }
  \newcommand{\modifica}[1]{%
    {\normalfont\bfseries\ttfamily modifica\ }%
    \ensuremath{##1}%
    {\normalfont\bfseries\,;\par}%
  }
  \renewcommand{\aux}[4]{%
    {\normalfont\bfseries\ttfamily aux\ }%
    {\normalfont\ttfamily ##1}%
    \ifthenelse{\equal{##2}{}}{}{\ (##2)}\ : ##3\, = \ensuremath{##4}%
    {\normalfont\bfseries\,;\par}%
  }
  \newcommand{\res}{#1}
  \vspace{1ex}
  \noindent
  \encabezadoDeProblema{#1}{#2}{#3}{#4}
  % Abrimos la llave
  \{\par%
  \tocarEspacios
}
% Ahora viene el cierre del ambiente problema
{
  % Cerramos la llave
  \noindent\}
  \vspace{1ex}
}

\newenvironment{tipo}[2][]{%
  % Preambulo del ambiente tipo
  % Tenemos que definir los comandos observador (con requiere) y aux
  \newcommand{\observador}[3]{%
    {\normalfont\bfseries\ttfamily observador\ }%
    {\normalfont\ttfamily ##1}%
    \ifthenelse{\equal{##2}{}}{}{\ (##2)}\ : ##3%
    {\normalfont\bfseries\,;\par}%
  }
  \newcommand{\requiere}[2][]{{%
    \addtolength{\leftskip}{3em}%
    \setlength{\parindent}{-2em}%
    {\normalfont\bfseries\ttfamily requiere}%
    \ifthenelse{\equal{##1}{}}{}{\ {\normalfont\ttfamily ##1} :}\ 
    \ensuremath{##2}%
    {\normalfont\bfseries\,;\par}}
  }
  \newcommand{\explicacion}[1][]{{%
    \addtolength{\leftskip}{3em}%
    \setlength{\parindent}{-2em}%
    \par \hspace{2.3em} ##1%
    }
  }
  \newcommand{\invariante}[2][]{%
    {\normalfont\bfseries\ttfamily invariante}%
    \ifthenelse{\equal{##1}{}}{}{\ {\normalfont\ttfamily ##1} :}\ 
    \ensuremath{##2}%
    {\normalfont\bfseries\,;\par}%
  }
  \renewcommand{\aux}[4]{%
    {\normalfont\bfseries\ttfamily aux\ }%
    {\normalfont\ttfamily ##1}%
    \ifthenelse{\equal{##2}{}}{}{\ (##2)}\ : ##3\, = \ensuremath{##4}%
    {\normalfont\bfseries\,;\par}%
  }
  \vspace{1ex}
  \noindent
  \encabezadoDeTipo{#1}{#2}
  % Abrimos la llave
  \{\par%
  \tocarEspacios
}
% Ahora viene el cierre del ambiente tipo
{
  % Cerramos la llave
  \noindent\}
  \vspace{1ex}
}

% Cuestiones de enunciados

% Primero definiciones de cosas al estilo title, author, date
\def\materia#1{\gdef\@materia{#1}}
\def\@materia{No especifi\'o la materia}
\def\lamateria{\@materia}

\def\cuatrimestre#1{\gdef\@cuatrimestre{#1}}
\def\@cuatrimestre{No especifi\'o el cuatrimestre}
\def\elcuatrimestre{\@cuatrimestre}

\def\anio#1{\gdef\@anio{#1}}
\def\@anio{No especifi\'o el anio}
\def\elanio{\@anio}

\def\fecha#1{\gdef\@fecha{#1}}
\def\@fecha{\today}
\def\lafecha{\@fecha}

\def\nombre#1{\gdef\@nombre{#1}}
\def\@nombre{No especific'o el nombre}
\def\elnombre{\@nombre}

\def\practica#1{\gdef\@practica{#1}}
\def\@practica{No especifi\'o el n\'umero de pr\'actica}
\def\lapractica{\@practica}

% Esta macro convierte el numero de cuatrimestre a palabras
\newcommand{\cuatrimestreLindo}{
  \ifthenelse{\equal{\elcuatrimestre}{1}}
  {Primer cuatrimestre}
  {\ifthenelse{\equal{\elcuatrimestre}{2}}
  {Segundo cuatrimestre}
  {Verano}}
}

\newcommand{\depto}{{UBA -- Facultad de Ciencias Exactas y Naturales --
      Departamento de Computaci\'on}}

\newcommand{\titulopractica}{
  \centerline{\depto}
  \vspace{1ex}
  \centerline{{\Large\lamateria}}
  \vspace{0.5ex}
  \centerline{\cuatrimestreLindo de \elanio}
  \vspace{2ex}
  \centerline{{\huge Pr\'actica \lapractica -- \elnombre}}
  \vspace{5ex}
  \arreglarincisos
  \newcounter{ejercicio}
  \newenvironment{ejercicio}{\stepcounter{ejercicio}\textbf{Ejercicio
      \theejercicio}%
    \renewcommand\@currentlabel{\theejercicio}%
  }{\vspace{0.2cm}}
}  

\newcommand{\titulotp}{
  \centerline{\depto}
  \vspace{1ex}
  \centerline{{\Large\lamateria}}
  \vspace{0.5ex}
  \centerline{\cuatrimestreLindo de \elanio}
  \vspace{0.5ex}
  \centerline{\lafecha}
  \vspace{2ex}
  \centerline{{\huge\elnombre}}
  \vspace{5ex}
}

% AMBIENTE CONSIGNAS
% Se usa en el TP para ir agregando las cosas que tienen que resolver
% los alumnos.
% Dentro del ambiente hay que usar \item para cada consigna

\newcounter{consigna}
\setcounter{consigna}{0}

\newenvironment{consignas}{%
  \newcommand{\consigna}{\stepcounter{consigna}\textbf{\theconsigna.}}%
  \newcommand{\ejercicio}[1]{\item ##1 }
  \renewcommand{\problema}[5][result]{\item
    \encabezadoDeProblema{##1}{##2}{##3}{##4}\hspace{1em}##5}%
  \newcommand{\invariante}[2]{\item%
    {\normalfont\bfseries\ttfamily invariante}%
    \ ##1\hspace{1em}##2%
  }
  \renewcommand{\aux}[4]{\item%
    {\normalfont\bfseries\ttfamily aux\ }%
    {\normalfont\ttfamily ##1}%
    \ifthenelse{\equal{##2}{}}{}{\ (##2)}\ : ##3 \hspace{1em}##4%
  }
  % Comienza la lista de consignas
  \begin{list}{\consigna}{%
      \setlength{\itemsep}{0.5em}%
      \setlength{\parsep}{0cm}%
    }
}%
{\end{list}}


% MACROS ESPECIFICAS DE IMPERATIVO

% El primer parametro es el nombre del segundo parametro del ==
% El segundo parametro es el nombre del tipo
\newcommand{\eligualigual}[2]%
{%
\begin{problema}{operator==}{this,#1 : #2}{Bool}
  \asegura{result == (this == #1)}
\end{problema}
}

% Manejo de listas
% LaTeX deja demasiado espacio entre los items para nuestros propósitos.
\renewcommand{\enumhook}{\setlength{\itemsep}{-4pt}}

% Esto ajusta el espacio que se deja antes y después de los
% environments multicol
\setlength{\multicolsep}{5pt}

\makeatother


\newcommand{\comen}[2]{%
\begin{framed}
\noindent \textsf{#1:} #2
\end{framed}
}

\begin{document}

\materia{Algoritmos y Estructura de Datos I}
\cuatrimestre{2}
\anio{2013}

\nombre{\LARGE TPE Cine}

\titulotp


\sinonimo{Actor}{String}
\sinonimo{Sala}{\ent}
\enumerado{género}{Aventura, Comedia, Drama, Romántica, Terror}

\vspace{0.5cm}

\begin{tipo}{Pelicula}
	\observador{nombre}{p: Pelicula}{String}
	\observador{géneros}{p: Pelicula}{[Género]}
	\observador{actores}{p: Pelicula}{[Actor]}
	\observador{es3D}{p: Pelicula}{Bool}
	\invariante[sinActoresRepetidos]{sinRepetidos(actores(p))}
	\invariante[sinGénerosRepetidos]{sinRepetidos(generos(p))}
	\invariante[génerosOrdenados]{generosOrd(generos(p))}
	\invariante[actoresOrdenados]{actoresOrd(actores(p))}
\end{tipo}

\begin{problema}{agruparPelisPorGenero}{ps:[Pelicula]}{[(Genero, [Pelicula])]}
	\asegura[generoSalidaEntrada]{(\forall  dupla \selec result)prm(dupla) 			\in obtenerGeneros(ps)}
	\asegura[pelisSalidaEntrada]{(\forall dupla \selec result, peli \selec 			sgd(dupla))peli \in ps}
	\asegura[pelisSalidaSinRepetir]{(\forall dupla \selec result)sinRepetidos(sgd(dupla))}
	\asegura[generosSalidaEnPelis]{(\forall dupla \selec result, peli \selec sgd(dupla)) prm(dupla) \in generos(peli)}
	\asegura[generosSalidaSinRepetir]{sinRepetidos(obtenerGenerosDupla(result))}
	\asegura[pelisEntradaEnSalida]{(\forall peli \selec ps, dupla \selec result, prm(dupla) \in generos(peli))peli \in sgd(dupla)}
	\asegura[generosEntradaEnSalida]{(\forall genero \selec obetenerGeneros(ps))genero \in obtenerGenerosDupla(result)}
\end{problema}

\begin{problema}{generarSagaDePeliculas}{as:[Actor], gs:[Genero], nombres:[String]}{[Pelicula]}{
	\asegura[mismosGeneros]{(\forall peli \selec result)mismos(generos(peli), elementosSinRepetir(gs))}
	\asegura[mismosActores]{(\forall peli \selec result)mismos(actores(peli), elementosSinRepetir(as))}
	\asegura[todosNombreEnNombres]{mismos(nombresDePeliculas(res), elementosSinRepetir(nombres))}
}
\end{problema}

\begin{tipo}{Ticket}
	\observador{película}{t: Ticket}{Pelicula}
	\observador{sala}{t: Ticket}{Sala}
	\observador{usado}{t: Ticket}{Bool}
\end{tipo}

\begin{problema}{películaMenosVista}{ts : [Ticket]}{\bool}
	\requiere[]{ts \neq \lv}
	\requiere[]{ticketsUsados(ts) \neq \lv}
	
	\asegura[peliEnTickets]{result \in peliculasVistas(ticketsUsados(ts))}
	\asegura[esLaPeliMenosVista]{\neg((\exists p \selec peliculasVistas(ticketsUsados(ts)))\newline ticketsPorPelicula(p, ticketsUsados(ts)) < ticketsPorPelicula(result, ticketsUsados(ts)))}
\end{problema}

\begin{problema}{todosLosTicketsParaLaMismaSala}{ts:[Ticket]}{\bool}
	\asegura[todosLosTicketsParaLaMismaSala]{res == (\forall t \selec [0\twodots|ts|-1))sala(ts_i) == sala(ts_{i+1})}
\end{problema}

\begin{problema}{cambiarSala}{ts:[Ticket], vieja: Sala, nueva: Sala}{}{}	
	\modifica{ts}
	\asegura [mismoNumeroTickets]{|ts| == |pre(ts)|}
	\asegura [cambioDeSala]{(\forall i \selec [0\twodots|pre(ts)|), sala(pre(ts)_i) == vieja) sala(ts_{i}) == nueva}
	\asegura [mismasPelis]{(\forall i \selec [0\twodots|pre(ts)|))pelicula(ts_{i}) == pelicula(pre(ts)_{i})}
	\asegura [ticketsSiguenUsados]{(\forall i \selec [0\twodots|pre(ts)|)usado(ts_i) == usado(pre(ts)_i)}
	\asegura[mismasSalas]{(\forall i \selec [0\twodots|pre(ts)|), sala(pre(ts)_i \neq vieja)sala(ts_i) == sala(pre(ts)_i)}
\end{problema}




\begin{tipo}{Cine}
	\observador{nombre}{c: Cine}{String}
	\observador{películas}{c: Cine}{[Peliculas]}
	\observador{salas}{c: Cine}{[Sala]}
	\observador{sala}{c: Cine, p: Pelicula}{Sala}
		\requiere{p \in peliculas(c)}
	\observador{espectadores}{c: Cine, s: Sala}{\ent}
		\requiere{s \in salas(c)}
	\observador{ticketsVendidosSinUsar}{c: Cine}{[Ticket]}

	\invariante[sinPeliculasRepetidas]{sinRepetidos(nombresDePeliculas(c))}
	\invariante[sinSalasRepetidas]{sinRepetidos(salas(c))}
	\invariante[salasDeCineSonSalas]{(\forall p \leftarrow peliculas(c)) sala(c,p) \in salas(c) }
	\invariante[espectadoresNoNegativos]{(\forall s \leftarrow salas(c)) espectadores(c,s) \geq 0 }
	\invariante[losTicketsVendidosEstanSinUsar]{(\forall t \selec ticketsVendidosSinUsar(c)) \lnot usado(t) }
	\invariante[salasConsistentes]{sinRepetidos(\comp{sala(c, peli)}{peli \selec peliculas(c)})}
	\invariante[losTicketsVendidosSonParaPeliculasDelCine]{(\forall t \selec ticketsVendidosSinUsar(c)) \newline pelicula(t) \in peliculas(c)\ \&\&\ sala(t) == sala(c, pelicula(t))}
\end{tipo}

\begin{problema}{cineVacio}{n: String}{Cine}
		\asegura[mismoNombre]{nombre(res) == n}
		\asegura[salasVacias]{salas(res) == \lv}
\end{problema}

\begin{problema}{agregarPelicula}{c: Cine, p: Pelicula, s: Sala}{}{}
	\requiere[esSala]{s \in salas(c)}
	\requiere[nuevaSalaVacia]{\neg((\exists pe \selec pelicula(c))sala(c, pe) == s)}
	\requiere[nuevaPelicula]{p \notin peliculas(c)}
	\modifica{c}
	\asegura[mismosNombres]{nombre(c) == nombre(pre(c))}
	\asegura[mismasPelis]{mismos(peliculas(pre(c)\masmas \ [p], peliculas(c))}
	\asegura[mismasSalas]{mismos(salas(c), salas(pre(c))}
	\asegura[peliEnSala]{sala(c,p)==s}
	\asegura[pelisEnMismasSalas]{(\forall p \selec peliculas(pre(c)))sala(c,p) == sala(pre(c), p)}
	\asegura[mismosEspectadores]{(\forall s \selec salas(c))espectadores(c,p) == espectadores(pre(c), p)}
	\asegura[mismosTickets]{mismos(ticketsVendidosSinUsar(c) == ticketsVendidosSinUsar(pre(c)))}
\end{problema}

\begin{problema}{cerrarSala}{c: Cine, s: Sala}{}{}
	\requiere[esSala]{ s \in salas(c)}
	\requiere[salaSinTickets]{(\forall t \selec ticketsVendidosSinUSar(c)) sala(t) \neq s}
	\modifica{c}
	\asegura[mismoNombre]{nombre(c) == nombre(pre(c))}
	\asegura[mismasSalas]{mismos(salas(c),\comp{sal}{sal \selec salas(pre(c)), sal \neq s})}
	\asegura[mismasPelis]{mismos(peliculas(c), \comp{peli}{peli \selec peliculas(pre(c)), peli \neq peliDeSala(s,c)})}
	\asegura[pelisEnMismasSalas]{(\forall p \selec peliculas(pre(c)))sala(c,p) == sala(pre(c), p)}
	\asegura[mismosEspectadores]{(\forall s \selec salas(c))espectadores(c,p) == espectadores(pre(c), p)}
	\asegura[mismosTickets]{mismos(ticketsVendidosSinUsar(c), ticketsVendidosSinUsar(pre(c)))}
\end{problema}

\begin{problema}{cerrarSalas}{c: Cine, e: \ent}{}{
\requiere {e \ge 0}
\modifica {c}
\asegura {seParecen(c,pre(c),e)}
}
\end{problema}

\begin{problema}{cerrarSalasDeLaCadena}{cs: [Cine], e: \ent}{}{}
	\requiere {e \ge 0}
	\modifica {cs}
	\asegura [mismosCines]{|cs| == |pre(cs)|}
	\asegura [cinesSeParecen]{(\forall c' \selec pre(cs))(\exists c \selec cs)seParecen(c, c`, e)}
\end{problema}

\begin{problema}{pelicula}{c: Cine, s: Sala}{Pelicula}
	\requiere[esSala]{s \in salas(c)}
	\requiere[tienePeli]{(\exists p \selec peliculas(c)) sala(c,p) == s}
	\asegura[esPeli]{result \in  peliculas(c)}
	\asegura[peliEnSala]{sala(c, result) == s}
\end{problema}

\begin{problema}{venderTicket}{c: Cine, p: Pelicula}{Ticket}
	\requiere[esPelicula]{p \in peliculas(c)}
	\modifica{c}
	\asegura[mismoNombre]{nombe(c) == nombre(pre(c))}
	\asegura[mismasSalas]{mismos(salas(pre(c)), salas(c))}
	\asegura[mismasPelis]{mismos(peliculas(pre(c)), peliculas(c))}
	\asegura[pelisEnMismasSalas]{(\forall pe \selec peliculas(pre(c))sala(pre(c), pe) == sala(c, pe)}
	\asegura[mismosEspectadores]{(\forall sa \selec salas(pre(c))espectadores(pre(c), pe) == espectadores(c, pe)}
	\asegura[mismosTickets]{mismos(ticketsVendidosSinUsar(pre(c)\masmas \ [result]), ticketsVendidosSinUsar(pre(c))}
\end{problema}

\begin{problema}{ingresarASala}{c: Cine, s: Sala, t: Ticket}{}
	\requiere[esSala]{s \in salas(c)}
	\requiere[salaTicketEsSala]{sala(t) == s}
	\requiere[ticketSinUsar]{t \in ticketsVendidosSinUsar(c)}
	\modifica{c}
	\modifica{t}
	\asegura[mismaPeli]{pelicula(t) == pelicula(pre(t))}
	\asegura[mismaSala]{sala(t) == sala(pre(t))}
	\asegura[ticketUsado]{usado(t)}
	\asegura[ingresaEspectador]{espectadores(c, s) == espectadores(pre(c), s) + 1}
	\asegura[mismasSalas]{mismos(salas(c), salas (pre (c)))}
	\asegura[mismoNombre]{nombre(c) == nombre(pre (c))}
	\asegura[mismasPelis]{mismos(peliculas (c), peliculas (pre (c)))}
	\asegura[pelisEnMismasSalas]{(\forall  p \selec peliculas (pre (c)))sala (c, p) == sala (pre(c), p)}
	\asegura[mismosEspectadores]{(\forall sa \selec salas (pre (c)), sa \neq s)espectadores (c, sa) == espectadores (pre(c), sa)}
	\asegura[mismosTickets]{mismos(ticketsVendidosSinUsar(c),\newline
		 \comp{ti}{ti \selec ticketsVendidosSinUsar(pre(c)),\newline ti \neq t})}
\end{problema}

\begin{problema}{pasarA3DUnaPelicula}{c: Cine, nombre: String}{Pelicula}{}
	\requiere[laPeliculaExiste]{(\exists p \selec peliculas(c))nombre(p) == nombre}
	\modifica{c}
	\asegura[mismoNombrePeli]{nombre(result) == nombre}
	\asegura[peli3D]{es3D(result)}
	\asegura[mismosGeneros]{generos(result) == generos(peliculaDeNombre(pre(c), nombre))}
	\asegura[mismosActores]{actores(result) == actores(peliculaDeNombre(pre(c), nombre))}
	\asegura[mismaSala]{sala(c, result) == sala (c, peliculaDeNombre(pre(c), nombre))}
	\asegura[mismoNombre]{nombre(c) == nombre(pre(c))}
	\asegura[mismasSalas]{salas(c) == salas(pre(c))}
	\asegura[mismasPelis]{mismos(nombrePelisDeCine(c),nombrePelisDeCine(pre(c)))}
	\asegura[pelisEnMismasSalas]{(\forall p \selec peliculas(c), nombre(p)\neq nombre) sala(c, p) == sala(pre(c), p)}
	\asegura[mismosEspectadores]{(\forall s \selec salas(c)) espectadores(c, s) == espectadores(pre(c), s)}
	\asegura[mismosTickets]{mismos(nombrePelisDeTicketsDeCine(c), nombrePelisDeTicketsDeCine(pre(c)))}

	
\end{problema}


\section{Auxiliares}
\aux{cuenta}{x: T, a: [T]}{\ent}{long(\comp{y}{y \selec a, y == x})}
\aux{mismos}{a, b: [T]}{\bool}{|a| == |b| \ \&\& \ (\forall c \in a)cuenta(c,a) == cuenta(c,b)}
\aux{sinRepetidos}{l : [T]}{\bool}{(\forall i, j \selec [0..|l|), i\neq j)l_i \neq l_j}
\aux{nombresDePeliculas}{c: Cine}{[String]}{[nombre(p) | p \selec peliculas(c)]}
\aux{tienePeli}{c: Cine, s: Sala}{\bool}{(\exists p \selec pelicula(c)) sala(c,p) = s}
\aux{obtenerGeneros}{ps: [Pelicula]}{[Genero]}{\comp{genero(p)}{p \selec ps}}
\aux{obtenerGenerosDupla}{x: [(Genero,[Pelicula])]}{[Genero]}{\comp{prm(dupla)}{dupla \selec x}}
\aux{generosOrd}{gs: [Genero]}{Bool}{(\forall i \selec [0..|gs|), i \neq |gs|)ord(gs_i) \le ord(gs_{i+1})}
\aux{actoresOrd}{as: [Actores]}{Bool}{(\forall i \selec [0..|as|), i \neq |gs|)ord(as_i) \le ord(as_{i+1})}
\aux{ticketsUsados}{s: [Ticket]}{[Ticket]}{\comp{x}{x \selec s, usado(x)}}
\aux{peliculasVistas}{s: [Ticket]}{[Pelicula]}{\comp{pelicula(x)}{x \selec s}}
\aux{ticketsPorPelicula}{p: Pelicula, tic: [Ticket]}{\ent}{\comp{t}{t \selec tic, pelicula(t) == p}}
\aux{elementosSinRepetir}{s: [T]}{[T]}{\comp{s_i}{i \selec [0..|s|), \neg(\exists j \selec [0..|s|), j \neq i) s_i == s_j}}
\aux{peliDeSala}{s: Sala, c: Cine}{Pelicula}{cab([peli | peli \selec peliculas(c), sala(c, peli) == s])}
\aux{salaSinTicketsVendidosConMasDe}{c: Cine, e \ent}{[Sala]}{[sala | sala \selec salas(c), (\forall t \selec ticketsVendidosSinUsar(c)) \newline
((sala \neq sala(t)) \&\& (espectadores(sala) \geq e) )]}
\aux{salasConMasDe}{c: Cine, e: \ent}{[Sala]}{[sala | sala \selec salas(c), espectadores(c, sala) \geq e]}
\aux{pelisDeSalas}{c: Cine, sal: [Sala]}{[Peliculas]}{[peliDeSala(c, s) | s \selec sal]}
\aux{mismoNombre}{c, c`: Cine}{\bool}{nombre(c) == nombre(c`)}
\aux{mismasSalas}{c, c`: Cine, e: \ent}{\bool}{mismos(salas(c), salasSinTicketsVendidosConMasDe(c`, e))}
\aux{mismasPeliculas}{c, c`: Cine, e: \ent}{\bool}{mismos(peliculas(c), pelisDeSalas(c`,salasConMasDe(c`, e)))}
\aux{pelisEnMismasSalas}{c, c`: Cine}{\bool}{(\forall p \selec peliculas(c)) sala(c, p)==sala(c`,p)}
\aux{mismosEspectadores}{c, c`: Cine}{\bool}{(\forall sala \selec salas(c))espectadores(c,sala)==espectadores(c`,sala)}
\aux{mismosTickets}{c, c`: Cine}{\bool}{mismos(ticketsVendidosSinUsar(c), ticketsVendidosSinUsar(c`))}
\aux{nombrePelisDeCine}{c: Cine}{[String]}{\comp{nombre(p)}{p\selec peliculas(c)}}
\aux{nombrePelisDeTicketsDeCine}{c: Cine}{[String]}{\comp{nombre(pelicula(t))}{t\selec ticketsVendidosSinUsar(c)}}
\aux{películaDeNombre}{c: Cine, n: String}{Pelicula}{\cab\comp{peli}{peli \selec peliculas(c), nombre(peli) == n}}
\aux{seParecen}{c, c`: Cine, e: \ent}{\bool}{\newline mismoNombre(c,c`) \&\&  \newline mismasSalas(c,c`,e) \&\& \newline mismasPeliculas(c,c`) \&\&  \newline pelisEnMismasSalas(c,c`) \&\& \newline mismosEspectadores(c,c`) \&\& \newline mismosTickets(c,c`)}


\section {Nota sobre los ejercicios 12, 13 y 14}
El argumento utilizado al momento de la resolución de estos problemas es que no es posible cerrar salas que esten relacionadas con aquellos tickets vendidos sin usar, ya que el invariante lo prohibe. Se podria haber optado tambien por eliminar los tickets de esas salas, pero fue considerado inapropiado, por lo tanto fue elegida la primera opción: no cerrar aquellas salas que tengan un ticket vendido sin usar asignado a ellas.

\end{document}
